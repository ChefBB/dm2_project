\section{Time Series Analysis}

The following chapter illustrates Time Series analysis.
This was done on a separate dataset, consisting of 1134 time series.
Each of them represented daily domestic box-office gross revenues in
the United States and Canada, spanning 100 days from release day (day
0) to day 99.
% The following analysis revolve around both time series-specific
% and more general techniques.
Each observation also includes descriptive metadata, adding
additional informations on titles.
% : \texttt{id},
% \texttt{genre} and \texttt{rating}.

\subsection{Data Understanding}

The dataset contains 104 attributes in  total: 100 numerical columns
corresponding to daily gross revenues, one numerical column for the
IMDb average \texttt{rating}, and three categorical columns
identifying the film (\texttt{id}, \texttt{genre}, and
\texttt{rating category}).
Preliminary inspection revealed no missing values.


% TODO: check
Descriptive statistics provide an overview of the box-office revenue trends. 
On release day, the average revenue is approximately 9 million USD,
with maximum values exceeding 150 million USD. 
Revenues decline rapidly in subsequent days, reaching mean values near
100000 USD by day 99.
Variance remains high across the series, indicating substantial
variability in revenue levels among films.

% \begin{figure}[H]
%     \centering
%     \includegraphics[width=0.65\textwidth]{plotsss/ts_distrib.png}
%     \caption{Log-scale box-office gross revenues}
%     \label{fig:ts_distrib}
% \end{figure}

\begin{figure}[ht]
    \centering
    \begin{minipage}{0.45\textwidth}
        \centering
        \includegraphics[width=\textwidth]{plotsss/ts_boxplot.png} 
        \captionof{figure}{Boxplot of box-office gross revenues}
        \label{fig:ts_boxplot} 
    \end{minipage}
    \hfill
    \begin{minipage}{0.5\textwidth} 
        \includegraphics[width=\textwidth]{plotsss/ts_distrib.png}
        \caption{Log-scale box-office gross revenues}
        \label{fig:ts_distrib}
    \end{minipage}

\end{figure}

Figure~\ref{fig:ts_boxplot} shows the boxplot distribution of all time series.
We can see a lot of fliers, especially in the first days after release,
indicating high variability in box-office revenues among different films.
All fliers are concentrated in the upper range, showing that data is
right-skewed, with a few films achieving exceptionally high revenues
compared to the majority. 

Figure~\ref{fig:ts_distrib} illustrates the log-scaled distribution of
all time series, grouped by rating class.
For each class, the mean time series is reported together with the
corresponding minimum and maximum values at each time step.
Logarithmic scale was adopted to better capture the strong seasonal
patterns present in the data, which are otherwise compressed in the
later time steps when using a linear scale.
From this plot, a downward trend is visible in all classes, showing
higher engagement at release.

Another important note on the dataset is that missing values for films with 
runs shorter than 100 time steps were completed through a synthetic extension
procedure.\\

Although a 7-step seasonal pattern appears to be present, the seasonality is 
much more pronounced at a 14-step interval. This suggests that the time series may 
consist of two observations per day, making a 14-step cycle correspond to a 
weekly pattern.
% This could be due to the fact that data were synthesized in between existing 
% observations.
This could indicate that the synthetic extensions may have been interleaved within 
the original time steps rather than appended only at the tail of the series. 
The insertion of noise-augmented mean values within the temporal sequence 
might alter the effective sampling structure and artificially reinforce 
longer-period seasonal patterns.
This interpretation seems to be consistent with the box office domain, where 
revenue peaks are typically observed during weekends, particularly around a 
movie’s release period. Moreover, in this context, 7 steps peaks would be
explained by midweek promotional activities.
\paragraph*{{Metadata}}
IMDb ratings show a mean of 6.6 with a standard deviation of 0.9.
The distribution ranged from 2.8 to 8.7.
The \texttt{rating\_category} variable is a binning of \texttt{rating},
useful for classification tasks in later sections.
The feature consists of five different classes:
\begin{itemize}
    % Low (10 titles), Medium Low (128), Medium (387), Medium High (232), and High (377). 
    \item \textbf{Low} (10 titles), representing ratings ranging from
    1.0 to 4.0;
    \item \textbf{Medium Low} (128 titles), representing ratings ranging from 4.1 to 5.5;
    \item \textbf{Medium} (387 titles), representing ratings ranging from 5.6 to 6.5;
    \item \textbf{Medium High} (232 titles), representing ratings from 6.6 to 7.0;
    \item \textbf{High} (377 titles), representing ratings from 7.1 to 10.0.
\end{itemize}
As can be seen, the distribution is highly imbalanced, with the Low
category being significantly underrepresented compared to the other
classes.



% TODO aggiungere parte di scaling eventuale o cose simili di preparation

\subsection{Motifs and Anomalies Discovery}



\section{Time Series Clustering}

The time series were first preprocessed using a logarithmic transformation
followed by per-series Z-score normalization. This ensures that each 
series contributes equally to the feature extraction and clustering, 
independent of its absolute scale.


\subsection{Baseline}
As a baseline, sequence of experiments was designed to systematically explore how different representations and distance measures
influenced the quality of time series clustering. 
We first evaluated simple Euclidean-based approaches using both agglomerative hierarchical clustering and \texttt{KMeans} 
in order to establish a straightforward baseline. 
We then introduced Dynamic Time Warping (DTW), 
%and repeated the experiments with agglomerative clustering and \texttt{KMeans}, 
since DTW is well suited for handling temporal misalignment and shape-based similarity. 
Finally, we extracted the \texttt{catch22} feature set and applied both clustering algorithms again to assess 
whether a compact, interpretable feature representation could outperform raw time series distances. 
This progression allowed us to compare distance-based versus feature-based methods, 
evaluate the impact of temporal alignment, and understand the trade-offs between computational cost
and interpretability across all clustering configurations.

In the agglomerative hierarchical clustering results, the dendrogram 
clearly indicated the presence of three dominant clusters. Based on this observation, to assess the robustness we experimented 
with cluster sizes of $k=3,4,5$ to examine how the \texttt{rating\_category} values were distributed across the groups. 
For \texttt{KMeans}, both the elbow method and the silhouette score suggested that $k=6$ was the most suitable choice, 
so to understand the stability of cluster structure we further inspected the distributions for $k=4,5,6$. Across these experiments, 
we observed that the majority of samples consistently formed one large, heterogeneous cluster containing 
almost all rating categories (\texttt{low}, \texttt{medium\_low}, \texttt{medium}, \texttt{medium\_high}, \texttt{high}), 
while a much smaller cluster captured only a few higher-rated series. 
Increasing the number of clusters mainly affected these small groups, which became further subdivided, 
whereas the large mixed cluster remained relatively stable. 
This behaviour was expected, as the IMDb time series exhibited strong overall similarity with 
only subtle differences in temporal patterns, causing most series to cluster together 
while only a few outliers separated.


% As a baseline, we applied K-Means clustering directly on the catch22 features, setting the number
% of clusters with elbow method and silhoutte score
% with k=2 and k=5()
% we can show dynamic time warping distane simple agglomerative clustering with average distance with k=2, or k=5
% we canshow simple agglomerative clustering with average distance with k=2, or k=5
% we can show simple agglomerative clustering with ward distance with k=2, or k=5
% we tried all these methods but due to page constraint we will show only the best one which is agglomerative clustering with average distance with k=5



% Pros and cons
% a) dynamic time warping takes too long even with sakoe chiba band as it is not useful, we use dtw 
% in the pipeline to reduce dimensionality
% b) simple agglomerative clustering with average distance and k = 2, we follow the algorithm, dendogram
% c) with k = 5, we make it comparable with k means with  catch 22 features
% d)

\subsection{Clustering Pipeline}
The following pipeline combines feature-based and shape-based clustering: 
initial feature extraction allows fast micro-clustering, 
while medoid selection and DTW-based final clustering focus on the 
temporal shape. 

\paragraph*{Feature extraction and micro-clustering.} 
From the normalized series, we extracted features using  
\texttt{catch22}.
These features were used for an initial micro-clustering step via 
\texttt{KMeans} with Euclidean distance setting k=200. The micro-clustering reduces 
the complexity of the dataset by grouping similar series in the feature 
space, enabling downstream operations on a smaller representative set.  

\paragraph*{Representative selection and PAA.} 

For each micro-cluster, we computed the medoid of the raw series using 
Dynamic Time Warping (DTW) distance. This approach preserves the temporal 
structure of the series within the cluster. The medoids were subsequently 
transformed using Piecewise Aggregate Approximation (PAA, $M=20$), which 
reduces the dimensionality while retaining the overall shape of the series. 
PAA also facilitates the use of DTW in the final clustering step by smoothing 
small fluctuations.  

\paragraph*{Final clustering.} 
Agglomerative clustering with average linkage was applied on the PAA-transformed 
medoids using DTW distance as the similarity measure. This allows grouping 
of series based on shape similarity rather than Euclidean proximity in feature space. 
The number of clusters was determined by visually inspecting the dendrogram and selecting a cut-off to obtain 5 clusters.
The final labels were propagated back to the original series.  

\paragraph*{Cluster analysis and visualization.} 
We analyzed the resulting clusters with respect to \texttt{rating\_category}. 
\textcolor{red}{write something about the distribution of rating categories across clusters.}
Figure~\ref{fig:ts_clustering_dist} shows the distribution of rating categories
across the clusters, while Figure~\ref{fig:ts_clustering_representatives}
displays the representative time series for each cluster.
To visualize the clusters, we projected the PAA-transformed medoids
into two dimensions using both PCA and t-SNE (perplexity=15).

\textcolor{red}{explain differences between pca and tsne results.}
% Figure~\ref{fig:cluster_2d} shows the resulting projections,
% colored by cluster membership. 


% \begin{figure}[h!]
%     \centering
%     \includegraphics[width=0.48\textwidth]{pca_clusters.png}
%     \includegraphics[width=0.48\textwidth]{tsne_clusters.png}
%     \caption{Two-dimensional projections of cluster representatives. \textbf{Left:} PCA. \textbf{Right:} t-SNE. Each color corresponds to a cluster.}
%     \label{fig:cluster_2d}
% \end{figure}

% Overall, the clusters reveal clear differences in temporal patterns among 
% the movies, particularly in peak timing, decay, and long-term tail behavior. 
% The similarity of results between PCA and t-SNE projections confirms that the cluster 
% structure is robust to the dimensionality reduction method. 

\begin{figure}[ht]
    \centering
    \begin{minipage}{0.5\textwidth}
        \centering
        \includegraphics[width=\textwidth]{plotsss/ts_clustering_dist.png} 
        \captionof{figure}{Rating category Distribution}
        \label{fig:ts_clustering_dist} 
    \end{minipage}
    \hfill
    \begin{minipage}{0.45\textwidth} 
        \includegraphics[width=\textwidth]{plotsss/ts_clustering_representatives.png}
        \caption{Final Cluster Representatives}
        \label{fig:ts_clustering_representatives}
    \end{minipage}


\end{figure}


\begin{figure}[ht]
    \centering
    \begin{minipage}{0.45\textwidth}
        \centering
        \includegraphics[width=\textwidth]{plotsss/ts_pca.png} 
        \captionof{figure}{PCA Projection}
        \label{fig:ts_pca} 
    \end{minipage}
    \hfill
    \begin{minipage}{0.45\textwidth} 
        \includegraphics[width=\textwidth]{plotsss/ts_tsne.png}
        \caption{t-SNE Projection}
        \label{fig:ts_tsne}
    \end{minipage}

\end{figure}

\subsection{Classification}

The classification task aims to predict the \texttt{rating\_category} of a film based on its daily box-office revenue time series.
This category originally had five classes: Low, Medium Low, Medium, Medium High, and High.
However, due to the significant class imbalance, with the Low category containing only 10 instances,
the decision was made to merge the Low and Medium Low categories into a single class.

\subsection{kNN Classifier}

%Preprocessing in rnn

The task was first addressed using the $k$-Nearest Neighbors (KNN) classifier with two 
different distance measures, namely Euclidean distance and Dynamic Time Warping (DTW), 
in order to compare a standard vector-based similarity with a time-aware distance specifically 
designed for temporal data. The input consisted exclusively of the raw time series, 
without additional metadata, in order to allow a direct comparison of the two distance metrics 
on the raw time series. A logarithmic transformation was first applied to reduce skewness and 
stabilize variance. For the Euclidean KNN, a global Z-score normalization was then performed 
across samples so that each time step contributed equally to the distance computation, 
while for the DTW-based KNN a per-series mean–variance normalization was applied to remove 
amplitude effects and focus the comparison on temporal shape rather than absolute scale. 
The dataset was split into stratified train and test sets (80/20), and \texttt{GridSearchCV} 
was used on the Euclidean KNN to optimize the number of neighbors setting as scoring \texttt{balanced\_accuracy}.
The best number of neighbors was found to be 9.

Using this configuration, the Euclidean KNN achieved an overall accuracy of approximately 
$0.46$, with a macro-averaged F1-score of $0.37$, indicating a moderate classification 
performance in a multi-class and partially imbalanced setting. The classification report
and the confusion matrix reveal
that the classifier is most effective in identifying the \textit{High} and \textit{Medium} 
rating categories, which also correspond to the most represented classes in the dataset. 
In particular, the \textit{High} class exhibits a relatively high recall (0.71),
indicating that highly rated items exhibit more distinctive temporal patterns.
In contrast, the \textit{Low} and \textit{Medium High} categories show substantially lower 
recall values, confirming that these categories are harder to separate and are often confused 
with adjacent rating levels. This behavior is consistent with the ordinal nature of the target variable, 
where errors tend to occur between semantically close categories rather than across 
distant ones.



The DTW-based KNN was evaluated using the same number of neighbors and a Sakoe–Chiba 
constraint with radius 7, chosen considering the observed seasonality of approximately 14 
time steps while preventing excessive temporal warping. The DTW classifier achieved a 
slightly lower accuracy of about $0.42$ and a macro F1-score of $0.35$, with performance trends similar to the Euclidean 
case, as shown in Figure~\ref{fig:cm_knn}. While DTW provides a more flexible alignment 
of temporal patterns, its advantage appears limited in this setting, likely due to the 
fact that, maybe, rating categories are more strongly associated with overall trend and magnitude 
patterns rather than fine-grained temporal misalignments. 


% Overall, the comparison suggests that, for this task, Euclidean KNN on globally normalized 
% time series offers a competitive and computationally simpler baseline, while DTW does not 
% yield a substantial improvement despite its higher computational cost.

\begin{figure}[t]
    \centering
    \begin{subfigure}{0.45\textwidth}
        \centering
        \includegraphics[width=\linewidth]{plotsss/cm_knn_euclidean.png}
        \caption{KNN with Euclidean distance}
    \end{subfigure}
    \hfill
    \begin{subfigure}{0.45\textwidth}
        \centering
        \includegraphics[width=\linewidth]{plotsss/cm_knn_dtw.png}
        \caption{KNN with DTW distance}
    \end{subfigure}
    \caption{Confusion matrices for the KNN classifiers using Euclidean and DTW distances.}
    \label{fig:cm_knn}
\end{figure}





\subsubsection{Shapelet-based Classification}
To move beyond global similarity measures and explicitly capture local
discriminative patterns in the revenue dynamics, the classification
task was also addressed using shapelet-based methods, using the 
extracted shapelet features to train a Random Forest classifier.\\

% This approach is particularly well suited to the present task, as
% films with similar overall revenue levels may still differ in
% localized behaviors such as early peaks, decay rates, or delayed
% growth.\\

The preprocessing pipeline first applied a logarithmic transformation
to the raw revenue time series to reduce skewness and stabilize
variance, followed by a per-series mean-variance normalization to
emphasize temporal shape over absolute scale. This choice is motivated
by the nature of shapelet-based methods, which aim to identify
discriminative local patterns whose relevance lies in their temporal
structure rather than in absolute revenue magnitude.

A \texttt{RandomShapeletTransform} was then fitted on the
training data to extract a fixed set of discriminative subsequences,
which were used to transform both training and test sets into feature
representations.
The minimum shapelet length was set to 7, while the maximum was set to
29. Other configurations were also tested, but were discarded because
of worse performances.\\

\begin{wraptable}{r}{0.35\textwidth}
    \centering
    \caption{Random Forest's overview}
    \label{tab:rf_best_params}
    \begin{tabular}{lc}
        \hline
        \textbf{Hyperparameter} & \textbf{Value} \\
        \hline
        \texttt{n\_estimators} & 100 \\
        \texttt{max\_depth} & 20 \\
        \texttt{min\_samples\_split} & 10 \\
        \texttt{min\_samples\_leaf} & 4 \\
        \texttt{max\_features} & \texttt{sqrt} \\
        \hline
        \textbf{Accuracy} & 0.46 \\
        \textbf{Macro F1} & 0.42 \\
        \hline
    \end{tabular}
\end{wraptable}

The Random Forest classifier used balanced class weights and was tuned
via a randomized hyperparameter search.
A \texttt{RandomizedSearchCV} procedure with 20 sampled configurations
and 3-fold cross-validation was employed, optimizing for balanced
accuracy to account for class imbalance.
Table~\ref{tab:rf_best_params} reports the best configuration, along
with key performance metrics.

The selected hyperparameters reflect a balanced model: sufficient trees
and depth for learning, while minimum sample thresholds and feature
subsampling help prevent overfitting.
The model achieves an overall accuracy of 46\%, consistent with the
previous model, while the macro-averaged F1-score is higher, indicating
improved handling of minority classes.

This is confirmed by the confusion matrix in
Figure~\ref{fig:cm_shapelets}.
The \textbf{Low} class is better captured relative to the kNN
classifiers. Errors for this and the \textbf{Medium High} class
are often close to the correct category.
The \textbf{Medium} and \textbf{High} classes are often
misclassified as each other, similar to the kNN models.
The \textbf{High} class performs slightly worse with respect to
the kNN models.

Figure~\ref{fig:shapelets} shows the eight shapelets with the highest
Information Gain. As indicated in the legend, the most informative
shapelets come from the \textbf{Low} and \textbf{Medium High} classes,
which are the least represented.
The top shapelet for the \textbf{High} class ranks eleventh overall,
while the top \textbf{Medium} class shapelet ranks sixteenth.

Notably, the shapelets vary considerably in length, shape, and position
within their respective time series, reflecting the diverse patterns the
model leverages for classification.


\begin{figure}[H]
    \centering
    \begin{minipage}{0.43\textwidth}
        \includegraphics[width=1\textwidth]{plotsss/cm_shapelets.png}
        \caption{Confusion matrix for the RF model based on shapelets}
        \label{fig:cm_shapelets}
    \end{minipage}
    \hfill
    \begin{minipage}{0.48\textwidth}
        \includegraphics[width=1.\textwidth]{plotsss/shapelets.png}
        \caption{8 most important shapelets}
        \label{fig:shapelets}
    \end{minipage}
\end{figure}


Figures~\ref{fig:toplow},~\ref{fig:topmedium},~\ref{fig:topmediumhigh},
and~\ref{fig:tophigh} show the most informative shapelets for the
Random Forest model, grouped by class.
Each shapelet is compared with the top four SAX motifs of the same
length, allowing a direct comparison of their shapes and illustrating
how discriminative subsequences relate to commonly recurring patterns
in the time series.

The shapelets for the \textbf{Low} and \textbf{High} classes generally
follow the shape of the corresponding motifs.
For the \textbf{Medium} class, the shape aligns with the \texttt{cbbb}
SAX motif, albeit shifted by a few time steps.
In contrast, the \textbf{Medium High} shapelet differs from all motifs
of that length; however, its form is consistent with parts of the extended
motifs identified in Section~\ref{subsec:motifs_discords}, suggesting
it likely still represents a non-anomalous pattern.


\begin{figure}[H]
    \centering
    \begin{minipage}{0.48\textwidth}
        \includegraphics[width=1\textwidth]{plotsss/toplow.png}
        \caption{Best shapelet for the Low class}
        \label{fig:toplow}
    \end{minipage}
    \hfill
    \begin{minipage}{0.48\textwidth}
        \includegraphics[width=1\textwidth]{plotsss/topmedium.png}
        \caption{Best shapelet for the Medium class}
        \label{fig:topmedium}
    \end{minipage}
\end{figure}
    
\begin{figure}[H]
    \centering
    \begin{minipage}{0.48\textwidth}
        \includegraphics[width=1\textwidth]{plotsss/topmediumhigh.png}
        \caption{Best shapelet for the Medium High class}
        \label{fig:topmediumhigh}
    \end{minipage}
    \hfill
    \begin{minipage}{0.48\textwidth}
        \includegraphics[width=1\textwidth]{plotsss/tophigh.png}
        \caption{Best shapelet for the High class}
        \label{fig:tophigh}
    \end{minipage}
\end{figure}


\subsubsection{Recurrent Neural Network}

\begin{wrapfigure}{r}{0.48\textwidth}
    \centering
    \includegraphics[width=\linewidth]{plotsss/rnn_model.png}
    \caption{Architecture of the RNN model}
    \label{fig:rnn_model}
\end{wrapfigure}

As the last model, a Recurrent Neural Network (RNN) was
implemented, because of the suitability of RNNs for
sequential data.
The preprocessing applied was log-scale transformation, followed
by global scaling, in order to preserve absolute scale.

Its architecture is shown in Figure~\ref{fig:rnn_model},
and was obtained through experimentation with
different configurations and hyperparameters.
The test and validation sets used 40\% of the dataset each,
while the training set used the remaining 60\%.
The split was stratified, to maintain class proportions across sets.

The genre features are handled the same way as in previous Neural
Networks (a Dense layer with 8 neurons and ReLU activation function,
the right-most branch in figure~\ref{fig:rnn_model}).

The time series data is processed through an encoder-decoder
architecture, to extract relevant features from the sequences.
The encoder processes the first half of the time series,
with two Bidirectional LSTM layers (with 32 and 64 units respectively).
The decoder processes the second half of the time series,
with two Bidirectional LSTM layers (with 64 and 32 units respectively).
The final shape resembles a diamond shape, with the number of units
first increasing, then decreasing, to capture both local and global
patterns in the data.
Both the encoder and decoder use recurrent dropout of 0.3 in every layer
for regularization.




The outputs of the two branches are finally concatenated, and
fed to a Dense layer with 64 neurons and ReLU activation function,
followed by a Dropout layer with rate 0.3.
The final output layer has 4 neurons (one for each class)
and a Softmax activation function.
Wherever possible, Batch Normalization was applied to speed up training
and improve stability.\\

The model was trained using the Adam optimizer, categorical
cross-entropy loss function and a learning rate of 0.005.
Early stopping was employed to halt training if the validation loss
did not improve for 20 consecutive epochs.
The model was trained for a maximum of 200 epochs with a batch size
of 32, considering balanced class weights.

Figure~\ref{fig:loss_acc} shows the evolution of training and
validation loss and accuracy over epochs.
\begin{figure}[H]
    \centering
    \includegraphics[width=1\textwidth]{plotsss/ts_loss_acc.png}
    \caption{Training and validation loss and accuracy over epochs}
    \label{fig:loss_acc}
\end{figure}

\begin{wrapfigure}{r}{0.4\linewidth}
    \centering
    \includegraphics[width=\linewidth]{plotsss/rnn_cm.png}
    \caption{Confusion matrix for the RNN model on the test set}
    \label{fig:cm_rnn}
\end{wrapfigure}

The loss curves show incremental improvement over epochs,
with relative stability.
Their shape do not indicate overfitting, as the validation loss
seems to find relative stability around the 50th epoch.
Accuracy curves seem to confirm that overfitting is not a central
issue.
They also show significant instability, likely due to the small
size of the dataset, which makes the small sizes of validation
and test sets more susceptible to changes in the model's behavior.

Figure~\ref{fig:cm_rnn} shows the confusion matrix for the RNN
model on the test set.

Both the \textbf{Low} and \textbf{Medium High} classes exhibit
weaker performance, likely due to their relatively small number
of samples. The use of balanced class weights appears to
increase confusion with adjacent classes, as the model assigns
similar importance to neighboring categories.

The \textbf{Medium} class is generally the most challenging to
distinguish.
It also accounts for the majority of non-adjacent
misclassifications, with instances frequently predicted as
\textbf{High}. This behavior suggests a substantial overlap
between the feature distributions of the \textbf{Medium}
class and the surrounding classes.

Table~\ref{tab:macro_weighted_avg} reports the macro- and
weighted-average precision, recall, and F1-score for the
RNN model.
Despite the limited dataset size and class imbalance, the
final model achieves a satisfactory accuracy.
Moreover, the macro and weighted averages indicate reasonably
consistent performance across all classes, without excessive
bias toward the most frequent ones.
    

\begin{table}[H]
    \centering
    \begin{tabular}{lccc}
        \toprule
        \textbf{Metric} & \textbf{Precision} & \textbf{Recall} & \textbf{F1-score} \\
        \midrule
        \textbf{Macro avg}    & 0.45 & 0.47 & 0.45 \\
        \textbf{Weighted avg} & 0.49 & 0.48 & 0.47 \\
        \midrule
        \textbf{Accuracy}     & & & 0.48 \\
        \bottomrule
    \end{tabular}
    \caption{Macro and weighted average precision, recall, and F1-score for the RNN model}
    \label{tab:macro_weighted_avg}
\end{table} 

\subsubsection{Model Comparison}

Overall, all approaches achieve comparable accuracy values, ranging
between 42\% and 48\%, reflecting the intrinsic difficulty of the
task and the strong overlap between rating classes.

The \textit{k-NN} classifiers provide a simple and interpretable baseline.
The Euclidean-distance variant slightly outperforms the DTW-based
version, suggesting that absolute trends and magnitude-related patterns
are more informative for rating prediction than fine-grained temporal
alignments.
Both variants struggle with minority classes and mainly confuse
adjacent rating levels, consistent with the ordinal nature of the
target.

The shapelet-based Random Forest achieves similar accuracy to the
\textit{k-NN} baseline but improves the macro-averaged F1-score,
indicating a better balance across classes.
This improvement is primarily driven by enhanced recognition of
underrepresented classes, particularly \textbf{Low}, at the cost of a
slight degradation in performance for the \textbf{High} class.

The RNN model attains the highest overall accuracy and competitive
macro- and weighted-average scores.
While its performance gains over classical methods are modest,
the RNN shows more consistent behavior across classes, without being
overly biased toward the majority ones.
However, training instability and residual confusion between
\textbf{Medium} and \textbf{High} classes suggest that model capacity
is constrained by the limited dataset size.
