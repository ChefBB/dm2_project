\section{Time Series Analysis}

\subsection{Data Understanding}

The dataset consists of 1134 films, each represented by a time series of daily domestic 
box-office gross revenues in the United States and Canada, spanning 100 days from release day (day 0) 
to day 99. Each observation also includes descriptive metadata. The dataset contains 104 attributes in 
total: 100 numerical columns corresponding to daily gross revenues, one numerical column for the IMDb average \texttt{rating}, 
and three categorical columns identifying the film \texttt{id}, \texttt{genre}, and \texttt{rating category}.

Preliminary inspection revealed no missing values. For films with runs shorter than 100 days, 
missing entries were completed through a synthetic extension procedure, which imputes values using a noise-augmented mean 
of the observed revenues. This ensures uniform series length, although it introduces artificial values that may influence 
analyses focusing on the later days of a film’s lifecycle.

Descriptive statistics provide an overview of the box-office revenue trends. 
On release day, the average revenue is approximately 9 million USD, with maximum values exceeding 150 million USD. 
Revenues decline rapidly in subsequent days, reaching mean values near 100000 USD by day 99. 
Variance remains high across the series, indicating substantial variability in revenue levels among films.
IMDb ratings show a mean of 6.6 with a standard deviation of 0.9, ranging from 2.8 to 8.7.
The \texttt{rating\_category} variable, which will serve as the target for the classification part, 
is distributed across five classes: Low (10 titles), Medium Low (128), Medium (387), Medium High (232), and High (377). 
The distribution is highly imbalanced, with the Low category being significantly underrepresented compared to the other classes.


The presence of extreme values and synthetic extensions may necessitate normalization to ensure comparability 
of the time series and support more robust analyses.


% aggiungere parte di scaling eventuale o cose simili di preparation

\subsection{Motifs and Anomalies Discovery}



\subsection{Clustering}

\subsection{Classification}