\section{Data Understanding and Preparation}
\label{sec:data_preparation}

The dataset contains 32 columns and 149531 rows of titles of different types.
For each title, the dataset contains information regarding many different
aspects. Table~\ref{tab:initial_categorical_features} lists the initial categorical
features.
\begin{table}[H]
    \centering
    \begin{tabular}{|p{4cm}|p{9cm}|}

        \hline
        \textbf{Feature} & \textbf{Description} \\ \hline
        % Categorical
        \texttt{originalTitle} & Original title \\ \hline
        \texttt{isAdult} & Whether or not the title is for adult \\ \hline
        \texttt{canHaveEpisodes} & Whether the title can have episodes \\ \hline
        \texttt{isRatable} & Whether the title can be rated by users \\ \hline
        \texttt{titleType} & Type of the title (e.g., movie, tvseries) \\ \hline
        \texttt{countryOfOrigin} & Countries where the title was primarily produced \\ \hline
        \texttt{genres} & Genres associated with the title \\ \hline
        \texttt{regions} & Regions for this version of the title \\ \hline
        \texttt{soundMixes} & Technical specification of sound mixes \\ \hline
        % Ordinal
        \texttt{worstRating} (ordinal) & Worst title rating \\ \hline
        \texttt{bestRating} (ordinal) & Best title rating \\ \hline
        \texttt{rating} (ordinal) & IMDB title rating class \\ \hline
    \end{tabular}
    \caption{Initial categorical features of the IMDb dataset}
    \label{tab:initial_categorical_features}
\end{table}


Of the initial categorical attributes, the following were removed:
\begin{itemize}
    \item \texttt{originalTitle}, as it did not provide particularly useful
    information;
    \item \texttt{isAdult}, as it was almost completely correlated with the
    \textit{Adult} genre, so a logical OR operation was performed, and the genre
    only was kept;
    \item \texttt{canHaveEpisodes}, as it was completely correlated with the title type
    being \textit{tvSeries} or \textit{tvMiniSeries};
    \item \texttt{isRatable}, as it was always true;
    \item \texttt{soundMixes}, as it required some domain knowledge to be understood, as well as having issues with the values it contained.
    \item \texttt{worstRating} and \texttt{bestRating}, as they were always
    1 and 10, respectively;
    \item \texttt{rating}, as it was obtainable from the \texttt{averageRating}
    continuous attribute, through a simple discretization.
\end{itemize}


Table~\ref{tab:initial_features_numerical} lists the initial numerical features.\\

Of the initial numerical attributes, the following were removed:
\begin{itemize}
    \item \texttt{endYear}, as it had no values for non-Series titles, and having around 50\% of missing values for \textit{tvSeries} and
\textit{tvMiniSeries};
    \item \texttt{numVotes}, as it had a very high correlation with \texttt{ratingCount}.
\end{itemize}


\begin{table}[H]
    \centering
    \begin{tabular}{|p{4cm}|p{9cm}|}
        \hline
        \textbf{Feature} & \textbf{Description} \\ \hline
        % Continuous
        \texttt{startYear} & Release year of the title (series start year for TV) \\ \hline
        \texttt{endYear} & TV Series end year \\ \hline
        \texttt{runtimeMinutes} & Primary runtime of the title, in minutes \\ \hline
        \texttt{numVotes} & Number of votes the title has received \\ \hline
        \texttt{numRegions} & Number of regions for this version of the title \\ \hline
        \texttt{totalImages} & Total number of images for the title \\ \hline
        \texttt{totalVideos} & Total number of videos for the title \\ \hline
        \texttt{totalCredits} & Total number of credits for the title \\ \hline
        \texttt{criticReviewsTotal} & Total number of critic reviews \\ \hline
        \texttt{awardWins} & Number of awards the title won \\ \hline
        \texttt{awardNominations} & Number of award nominations excluding wins \\ \hline
        \texttt{ratingCount} & Total number of user ratings submitted \\ \hline
        \texttt{userReviewsTotal} & Total number of user reviews \\ \hline
        \texttt{castNumber} & Total number of cast individuals \\ \hline
        \texttt{CompaniesNumber} & Total number of companies that worked for the title \\ \hline
        \texttt{averageRating} & Weighted average of all user ratings \\ \hline
        \texttt{externalLinks} & Total number of external links on IMDb page \\ \hline
        \texttt{quotesTotal} & Total number of quotes on IMDb page \\ \hline
        \texttt{writerCredits} & Total number of writer credits \\ \hline
        \texttt{directorCredits} & Total number of director credits \\ \hline
    \end{tabular}
    \caption{Initial numerical features of the IMDb dataset}
    \label{tab:initial_features_numerical}
\end{table}



\subsection{Exploratory Data Analysis}

\begin{wrapfigure}{r}{0.4\textwidth}
    \centering
    \includegraphics[width=0.38\textwidth,height=4.5cm]{plotsss/rating_distrib.png}
    \caption{Distribution of the \texttt{averageRating}}
    \label{fig:rating_dist}
\end{wrapfigure}

To gain an initial understanding of the dataset, 
we conducted several exploratory analyses focusing on key attributes related to ratings, genres, 
and temporal trends. 
\newline Figure~\ref{fig:rating_dist} shows the distribution of the \texttt{averageRating} attribute, 
which appears approximately Normal with a peak around 7. This feature is central to both the classification and regression tasks, 
making its distribution essential for understanding model behavior and potential biases.


Figure~\ref{fig:genre_popularity} shows the distribution of titles across different genres.  
The chart reveals that nearly half of the titles fall under the Drama and Comedy genres, 
highlighting their dominant role in the entertainment industry and audience preferences.
% This insight helps guide further analysis, 
% such as predicting genre popularity, tailoring recommendations, or identifying gaps in content diversity.\\


To understand viewer perceptions we plotted figure~\ref{fig:year_rating} the trend of average ratings over time.
It reveals historical shifts in audience engagement 
and rating standards. We observe a steady increase in average ratings from the early 1900s to the present, 
with notable peaks around the 1960s and 2020s. 

We observed that several features in the dataset exhibited a heavy right skew, 
characterized by a long tail of high values. Based on this insight, we proceeded to the feature engineering phase, 
where we merged relevant columns and applied preprocessing techniques such as one-hot encoding and normalization to 
prepare the data for modeling.


\begin{figure}[H]
    \centering
    \begin{minipage}[t]{0.48\textwidth}
        \centering
        \includegraphics[width=\linewidth]{plotsss/genre_popularity.png}
        \caption{Distribution of titles across genres}
        \label{fig:genre_popularity}
    \end{minipage}
    \hspace{0.02\textwidth}
    \begin{minipage}[t]{0.48\textwidth}
        \centering
        \includegraphics[width=\linewidth]{plotsss/year_rating.png}
        \caption{Average rating trend over the years}
        \label{fig:year_rating}
    \end{minipage}
\end{figure}


\subsection{Feature Engineering}
\begin{itemize}
    \item \texttt{totalImages}, \texttt{totalVideos} and \texttt{quotesTotal} were merged
    through a simple sum operation
    into\newline\texttt{totalMedia} because of their similar semantic
    meaning;
    
    \item \texttt{awardWins} and \texttt{awardNominations} were merged
    through a simple sum operation into \texttt{totalNominations};

    \item \texttt{userReviewsTotal} and \texttt{criticReviewsTotal} were merged
    through a simple sum operation into\newline\texttt{reviewsTotal};

    \item \texttt{regions} and \texttt{countryOfOrigin}
    were merged through a simple union operation. The resulting feature was then
    represented trhough frequency encoding on the entire list, as well as
    counts of the number of countries from each continent.
    This resulted in eight new features (six continents, one for unknown country codes,
    and the last for the frequency encoding);

    \item \texttt{genre} each record
    contained up to three genres, listed in alphabetical order, indicating that the order
    did not convey any semantic information about the title.
    To represent this information, three separate features were created, each
    corresponding to one of the genres. These features were encoded using frequency
    encoding.  
    A value of 0 was used to indicate missing genres;

    \item \texttt{titleType} was one-hot encoded, as it was a nominal categorical
    feature with no intrinsic order;
    
    \item \texttt{deltaCredits} was created as the difference between
    \texttt{totalCredits} and the sum of \texttt{castNumber},\newline\texttt{writerCredits} and \texttt{directorCredits}. 
    This feature aimed to capture the number of other types of credits
    (such as producers, editors, etc.) associated with a title;
    
    \item \texttt{runtimeMinutes} had $\approx$27\% missing values. 
    Since the feature had high relevance in the domain, it was
    imputed with random numbers within the interquartile range, separately for each title
    type. For advanced classification, in cases where the target variable
    was \texttt{titleType}, the feature was imputed
    within the global interquartile range instead, in order
    to avoid leakage.

\end{itemize}