\subsection{Time Series Clustering}

The time series were first preprocessed using a logarithmic transformation
followed by per-series Z-score normalization. This ensured that
the clustering process revolved around shapes of the time series,
rather than their absolute scale.


\subsubsection{Baseline}
As a baseline, a sequence of experiments were designed to systematically
explore how different representations and distance measures
influenced the quality of time series clustering. 
Only the two that were considered to be more appropriate were reported.

An \textit{Agglomerative Hierarchical Clustering} was performed,
using \textit{Dynamic Time Warping} (DTW) as distance metric.
Instead of calculating all distances, a Sakoe-Chiba band
with radius of 10\% (10 time steps) was applied.
The selected linkage method was \textit{Average Linkage}.
After analyzing the dendrogram, shown in figure~\ref{fig:dtw_agglo},
it was cut at three clusters.

In order to explore clustering based on structural features,
we extracted the \texttt{catch22} feature set and applied \textit{K-Means}
to see whether a compact, interpretable feature representation could also
provide interesting insights.
% In the \textit{Agglomerative hierarchical clustering} results, the
% dendrogram figure~\ref{fig:dtw_agglo} clearly indicated the presence of
% three dominant clusters.
% Based on this observation, to assess the robustness we also experimented 
% with cluster sizes of $k=3,4,5$.
%to examine how the \texttt{rating\_category} values were distributed across the groups. 
The elbow method (figure~\ref{fig:elbow_kmean}) and the silhouette score
(figure~\ref{fig:silhouette}) suggested that $k=6$ was the most
suitable choice.


For the first method, it was observed that the majority of samples
consistently formed one large, heterogeneous cluster containing 
almost all rating categories,
while the two smaller ones focused almost exclusively on higher-rated
series.

For the second method, the obtained clusters were more balanced,
and clusters were generally heterogeneous.



\begin{figure}[htbp]
    \centering
    \begin{minipage}[t]{0.32\textwidth}
        \centering
        \includegraphics[width=\linewidth]{plotsss/dtw_agglo.png}
        \caption{Agglomerative Clustering Dendrogram}
        \label{fig:dtw_agglo}
    \end{minipage}
    \hfill
    \begin{minipage}[t]{0.32\textwidth}
        \centering
        \includegraphics[width=\linewidth]{plotsss/elbow_kmean.png}
        \caption{K-Means Elbow Method}
        \label{fig:elbow_kmean}
    \end{minipage}
    \hfill
    \begin{minipage}[t]{0.32\textwidth}
        \centering
        \includegraphics[width=\linewidth]{plotsss/silhouette.png}
        \caption{K-Means Silhouette Score}
        \label{fig:silhouette}
    \end{minipage}
\end{figure}

\subsubsection{Clustering Pipeline}
The following pipeline combines feature-based and shape-based clustering: 
initial feature extraction allows fast micro-clustering, 
while medoid selection and DTW-based final clustering focus on the 
temporal shape. 

\paragraph*{Feature extraction and micro-clustering.} 
From the normalized series, we extracted features using  
\texttt{catch22}.
These features were used for an initial micro-clustering step via 
\texttt{KMeans} with Euclidean distance setting k=200. The micro-clustering reduces 
the complexity of the dataset by grouping similar series in the feature 
space, enabling downstream operations on a smaller representative set.  

\paragraph*{Representative selection and PAA.} 

For each micro-cluster, we computed the medoid of the raw series using 
Dynamic Time Warping (DTW) distance. This approach preserves the temporal 
structure of the series within the cluster. The medoids were subsequently 
transformed using Piecewise Aggregate Approximation (PAA, $M=20$), which 
reduces the dimensionality while retaining the overall shape of the series. 
PAA also facilitates the use of DTW in the final clustering step by smoothing 
small fluctuations.  

\paragraph*{Final clustering.} 
Agglomerative clustering with average linkage was applied on the PAA-transformed 
medoids using DTW distance as the similarity measure. This allows grouping 
of series based on shape similarity rather than Euclidean proximity in feature space. 
The number of clusters was determined by visually inspecting the dendrogram and selecting a cut-off to obtain 5 clusters.
The final labels were propagated back to the original series.  \\

%\paragraph*{Cluster analysis and visualization.} 
We analyzed the resulting clusters with respect to \texttt{rating\_category} metadata. 
Figure~\ref{fig:ts_clustering_dist} shows the distribution of rating categories
across the clusters, while Figure~\ref{fig:ts_clustering_representatives}
displays the representative time series for each cluster.

% Figure~\ref{fig:cluster_2d} shows the resulting projections,
% colored by cluster membership. 

% \begin{figure}[h!]
%     \centering
%     \includegraphics[width=0.48\textwidth]{pca_clusters.png}
%     \includegraphics[width=0.48\textwidth]{tsne_clusters.png}
%     \caption{Two-dimensional projections of cluster representatives. \textbf{Left:} PCA. \textbf{Right:} t-SNE. Each color corresponds to a cluster.}
%     \label{fig:cluster_2d}
% \end{figure}

% Overall, the clusters reveal clear differences in temporal patterns among 
% the movies, particularly in peak timing, decay, and long-term tail behavior. 
% The similarity of results between PCA and t-SNE projections confirms that the cluster 
% structure is robust to the dimensionality reduction method. 

\begin{figure}[ht]
    \centering
    \begin{minipage}[t]{0.55\textwidth}
        \centering
        \includegraphics[width=\textwidth]{plotsss/ts_clustering_dist.png} 
        \captionof{figure}{Rating category Distribution}
        \label{fig:ts_clustering_dist} 
    \end{minipage}
    \hfill
    \begin{minipage}[t]{0.43\textwidth} 
        \includegraphics[width=\textwidth]{plotsss/ts_clustering_representatives.png}
        \captionof{figure}{Final Cluster Representatives}
        \label{fig:ts_clustering_representatives}
    \end{minipage}

\end{figure}


Cluster~3 is the largest and most heterogeneous group, containing movies
from all rating categories, with a predominance of medium-rated titles.
The shape of its representative time series is characterized by very high
opening revenues followed by a monotonic decline.
Clusters~0,~1 and~4 are instead dominated by highly rated movies and 
Cluter~1 has also a moderate amount of medium low rating movies.
Cluster 0 is characterized by a pronounced peak around the midpoint of the observation window,
Cluster 4 starts higher and has a similar overall shape to Cluster~0 but shifted toward earlier
time steps.
Cluster 1 displays an overall flatter temporal profile.
Finally, Cluster~2 consists of a single highly rated movie and captures an
atypical temporal pattern, with very low opening revenues followed by a sharp
increase toward the end of the observation window, highlighting an outlier
behavior.

To visualize the cluster structure, the 200 PAA-transformed medoids 
were projected into two dimensions using both PCA and t-SNE (perplexity=15). 
Both techniques reveal a modest separation among the three most populated clusters. 
In particular, the t-SNE projection highlights a clearer separation of Cluster~3 
from the others.

\begin{figure}[H]
    \centering
    \begin{minipage}{0.45\textwidth}
        \centering
        \includegraphics[width=\textwidth]{plotsss/ts_pca.png} 
        \captionof{figure}{PCA Projection}
        \label{fig:ts_pca} 
    \end{minipage}
    \hfill
    \begin{minipage}{0.45\textwidth} 
        \includegraphics[width=\textwidth]{plotsss/ts_tsne.png}
        \caption{t-SNE Projection}
        \label{fig:ts_tsne}
    \end{minipage}

\end{figure}





\begin{table}[H]
    \centering
    \caption{Descriptive statistics of box office Gross by cluster.}
    \label{tab:cluster_descriptive}
    \begin{tabular}{ccccccccc}
    \toprule
    Cluster & Size & Mean & Std & Min & Max & Start ($t=0$) & Mid ($t=50$) & End ($t=99$) \\
    \midrule
    0 & 87  & $5.62\times10^{5}$ & $1.67\times10^{6}$ & $80$ & $3.45\times10^{7}$ & $1.49\times10^{5}$ & $8.32\times10^{5}$ & $2.07\times10^{5}$ \\
    1 & 5   & $2.03\times10^{5}$ & $3.11\times10^{5}$ & $49$ & $2.38\times10^{6}$ & $2.50\times10^{5}$ & $2.92\times10^{5}$ & $5.77\times10^{4}$ \\
    2 & 1   & $4.75\times10^{5}$ & $8.62\times10^{5}$ & $4834$ & $3.99\times10^{6}$ & $6.95\times10^{3}$ & $4.99\times10^{4}$ & $5.10\times10^{5}$ \\
    3 & 953 & $1.63\times10^{6}$ & $4.36\times10^{6}$ & $8$ & $1.57\times10^{8}$ & $1.06\times10^{7}$ & $6.42\times10^{5}$ & $1.03\times10^{5}$ \\
    4 & 88  & $7.19\times10^{5}$ & $1.93\times10^{6}$ & $4$ & $2.41\times10^{7}$ & $1.48\times10^{6}$ & $6.36\times10^{5}$ & $1.07\times10^{5}$ \\
    \bottomrule
    \end{tabular}
\end{table}



    % --- LEFT: plot ---
    \begin{wrapfigure}{l}{0.5\textwidth}
        \centering
        \vspace{-0.5em}
        \includegraphics[width=\linewidth]{plotsss/ts_log_clustering.png}
        \caption{Log-scale box office gross by cluster.}
        \label{fig:cluster_log}
        \vspace{-1em}
    \end{wrapfigure}
    \hfill
    
   
    Table~\ref{tab:cluster_descriptive} reports descriptive statistics of the domestic
    box office gross time series for each cluster.
    In addition to global statistics (mean, standard deviation, minimum and maximum),
    the average gross at the release time step ($t=0$), at the midpoint of the observation
    window ($t=50$), and at the final time step ($t=99$) is reported in order to summarize
    the typical temporal evolution of movies belonging to each cluster.
    These pieces of information about absolute values are complemented by
    the log-scale plot in Figure~\ref{fig:cluster_log}, which provides an overview
    of the temporal dynamics for all movies within each cluster. 
    Solid lines represent the average trajectory within each cluster, while shaded areas indicate 
    the variability across movies, measured as the range between minimum and maximum values at each time step.
    Compared to the previous analysis on the representative microcluster medoids, 
    now the full set of original time series is considered, confirming that the
    temporal patterns identified earlier are generally preserved.
    Cluster~2, originally identified as a single outlier among the medoids, 
    remains an isolated series after mapping back to the original time series,
    confirming its atypical behavior.

    Even if 5 clusters were retrieved, clusters 1 and 5 could be
    considered outliers. This provides a direct link with the baseline
    hierarchical clustering. The rating distribution of the
    clusters is also comparable, although the cluster sizes are quite
    different.

    % The results highlight substantial heterogeneity across clusters, both in terms of
    % revenue scale and temporal dynamics.
    % The largest cluster (Cluster~3, 953 movies) is characterized by extremely high
    % initial revenues, with an average gross exceeding $10^7$ USD at release,
    % followed by a sharp decline over time.
    % A large fraction of total revenue is therefore concentrated in the initial period
    % after release.
    
    % Clusters~4 and~0 (88 and 87 movies, respectively) exhibit intermediate behaviors.
    % Cluster~4 shows relatively high initial revenues (on the order of $10^6$ USD)
    % followed by a gradual decay, whereas Cluster~0 is characterized by lower initial values
    % and a pronounced peak around the midpoint of the observation window.
    
    % The remaining clusters are much smaller in size and capture atypical dynamics.
    % Cluster~1 includes movies with consistently low revenues throughout the observation
    % period, while Cluster~2 consists of a single movie displaying an unusual pattern
    % with very low opening revenues and a sharp increase toward the end of the time horizon,
    % indicative of outlier behavior.



