\section{Time Series Clustering}

The time series were first preprocessed using a logarithmic transformation
followed by per-series Z-score normalization. This ensures that each 
series contributes equally to the feature extraction and clustering, 
independent of its absolute scale.  
The following pipeline combines feature-based and shape-based clustering: 
initial feature extraction allows fast micro-clustering, 
while medoid selection and DTW-based final clustering focus on the 
temporal shape. 

\paragraph*{Feature extraction and micro-clustering.} 
From the normalized series, we extracted features using  
\texttt{catch22}.
These features were used for an initial micro-clustering step via 
\texttt{KMeans} with Euclidean distance setting k=200. The micro-clustering reduces 
the complexity of the dataset by grouping similar series in the feature 
space, enabling downstream operations on a smaller representative set.  

\paragraph*{Representative selection and PAA.} 

For each micro-cluster, we computed the medoid of the raw series using 
Dynamic Time Warping (DTW) distance. This approach preserves the temporal 
structure of the series within the cluster. The medoids were subsequently 
transformed using Piecewise Aggregate Approximation (PAA, $M=20$), which 
reduces the dimensionality while retaining the overall shape of the series. 
PAA also facilitates the use of DTW in the final clustering step by smoothing 
small fluctuations.  

\paragraph*{Final clustering.} 
Agglomerative clustering with average linkage was applied on the PAA-transformed 
medoids using DTW distance as the similarity measure. This allows grouping 
of series based on shape similarity rather than Euclidean proximity in feature space. 
The number of clusters was determined by visually inspecting the dendrogram and selecting a cut-off to obtain 5 clusters.
The final labels were propagated back to the original series.  

\paragraph*{Cluster analysis and visualization.} 
We analyzed the resulting clusters with respect to \texttt{rating\_category}. 
\textcolor{red}{write something about the distribution of rating categories across clusters.}
Figure~\ref{fig:ts_clustering_dist} shows the distribution of rating categories
across the clusters, while Figure~\ref{fig:ts_clustering_representatives}
displays the representative time series for each cluster.
To visualize the clusters, we projected the PAA-transformed medoids
into two dimensions using both PCA and t-SNE (perplexity=15).

\textcolor{red}{explain differences between pca and tsne results.}
% Figure~\ref{fig:cluster_2d} shows the resulting projections,
% colored by cluster membership. 


% \begin{figure}[h!]
%     \centering
%     \includegraphics[width=0.48\textwidth]{pca_clusters.png}
%     \includegraphics[width=0.48\textwidth]{tsne_clusters.png}
%     \caption{Two-dimensional projections of cluster representatives. \textbf{Left:} PCA. \textbf{Right:} t-SNE. Each color corresponds to a cluster.}
%     \label{fig:cluster_2d}
% \end{figure}

% Overall, the clusters reveal clear differences in temporal patterns among 
% the movies, particularly in peak timing, decay, and long-term tail behavior. 
% The similarity of results between PCA and t-SNE projections confirms that the cluster 
% structure is robust to the dimensionality reduction method. 

\begin{figure}[ht]
    \centering
    \begin{minipage}{0.5\textwidth}
        \centering
        \includegraphics[width=\textwidth]{plotsss/ts_clustering_dist.png} 
        \captionof{figure}{Rating category Distribution}
        \label{fig:ts_clustering_dist} 
    \end{minipage}
    \hfill
    \begin{minipage}{0.45\textwidth} 
        \includegraphics[width=\textwidth]{plotsss/ts_clustering_representatives.png}
        \caption{Final Cluster Representatives}
        \label{fig:ts_clustering_representatives}
    \end{minipage}


\end{figure}


\begin{figure}[ht]
    \centering
    \begin{minipage}{0.45\textwidth}
        \centering
        \includegraphics[width=\textwidth]{plotsss/ts_pca.png} 
        \captionof{figure}{PCA Projection}
        \label{fig:ts_pca} 
    \end{minipage}
    \hfill
    \begin{minipage}{0.45\textwidth} 
        \includegraphics[width=\textwidth]{plotsss/ts_tsne.png}
        \caption{t-SNE Projection}
        \label{fig:ts_tsne}
    \end{minipage}

\end{figure}