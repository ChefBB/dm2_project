\subsection{Motifs and Anomalies Discovery}\label{subsec:motifs_discords}

To analyze temporal patterns in the movie time series an
approach based on Matrix Profile analysis was adopted.
Prior to motifs and discords extraction, all series were normalized 
using z-score normalization applied independently to each time series, 
in order to focus the analysis on pattern shape rather than absolute 
scale. Motifs were extracted using the \texttt{STUMPY} implementation 
of the matrix profile.
% A sliding window of length 28 was selected as a compromise between capturing meaningful temporal 
% dynamics and preserving local variations. 
A window length of 28 was selected to capture repeated seasonal 
dynamics rather than isolated cycles.

% Figure~\ref{fig:matrix_profile} shows an example of the motifs and
% discords detection, on time series 1000, with \texttt{ratingCategory}
% equal to High. Because of the relative length of the detected patterns
% with respect to the total length of the time series, only the lowest
% (motif) and the highest (discord) values for the matrix profile
% are highlighted.
% As can be seen in this example, the motif captures a shape that is
% encountered frequently, while the discord highlights a pattern that almost
% seems random.


Figure~\ref{fig:matrix_profile} shows an example of motif and discord
detection for time series 1000, whose \texttt{ratingCategory} is
\textbf{High}.
Due to the relative length of the detected subsequences with respect to
the full time series, only the global minimum (motif) and maximum
(discord) values of the matrix profile are highlighted. 

In this example, the motif captures a shape that occurs repeatedly
within the series, whereas the discord corresponds to a highly
irregular subsequence, exhibiting little resemblance to the remaining
patterns in the time series.

Complementary analyses indicated that the identified discords did not
exhibit clear or consistent structural patterns.
This was done both by exploring single time series, and by aggregating
the retrieved patterns through clustering.
Therefore, the following analyses do not take them into account, and
instead focused solely on motifs.



\begin{figure}[H]
    \centering
    \includegraphics[width=0.8\textwidth]{plotsss/motif_discord.png}
    \caption{Example of Matrix Profile on Time Series 1000 (\texttt{ratingCategory} = High)} 
    \label{fig:matrix_profile}
\end{figure}

For all following motifs-related analyses, the three subsequences
associated with the 
lowest matrix profile values were selected, resulting in the extraction 
of the top-3 motifs per time series. Furthermore, discords were not
considered to be insightful enough
All extracted motifs were aggregated into 
a single dataset. To analyze similarities among 
motifs, K-Means clustering using Euclidean distance was applied. 
The number of clusters was selected by jointly evaluating the Sum 
of Squared Errors and the Silhouette Score, leading to the choice 
of four clusters as a reasonable trade-off between compactness and 
separation. Cluster centroids were analyzed as representative 
average patterns, and intra-cluster variability was quantified 
by computing the standard deviation across all motifs assigned to 
each cluster. Figure~\ref{fig:motif_clusters} shows the resulting 
cluster centroids together with their variability bands. In addition, 
clustering was also performed using Dynamic Time Warping as distance 
measure in order to account for possible temporal misalignments 
between motifs; this alternative formulation led to cluster 
structures and representative patterns that were largely consistent 
with those obtained using Euclidean distance. \\

\begin{figure}[H]
    \centering
    \begin{minipage}{0.45\textwidth}
        \centering
        \includegraphics[width=\textwidth]{plotsss/ts_motifs_kmeans.png} 
        \captionof{figure}{Motif clusters via K-Means.}
        \label{fig:motif_clusters}
    \end{minipage}
    \hfill
    \begin{minipage}{0.45\textwidth} 
        \includegraphics[width=\textwidth]{plotsss/ts_sax.png}
        \caption{SAX representation of motifs.}
        \label{ fig:sax_representation }
    \end{minipage}

\end{figure}

The two plots reveal three partially overlapping groups of motifs,
whose shapes are characterized by wave-like patterns with evolving
peaks:

\begin{itemize}
    \item \textbf{Cluster 0} (\textbf{cccb}) predominantly captures
    motifs from the earliest stages of the time series, characterized
    by a high initial peak followed by a gradual decline;

    \item \textbf{Cluster 2} (\textbf{cbbb}) corresponds mainly to
    early-to-middle stages of the time series, exhibiting moderate
    peaks and a smoother overall wave shape;

    \item \textbf{Cluster 3} (\textbf{baaa}) represents motifs from
    the mid-to-late stages of the time series, with a noticeably
    flattened waveform and lower-amplitude peaks.
\end{itemize}

The remaining SAX representation depicts motifs occurring in
later stages of the time series, which display an almost flat shape.
These motifs are therefore also assigned to \textbf{Cluster 3},
reinforcing its interpretation as capturing late-stage, low-variability
patterns.

The final cluster (\textbf{Cluster 1}), which is also the least
populated, exhibits a distinct behavior: its motifs show a clear
increasing trend, with rising peaks over time, setting it apart from
the predominantly decaying or flattening patterns observed in the
other clusters.