\subsection{Motifs and Anomalies Discovery}

\hl{To analyze recurrent temporal patterns in the movie time series, 
we adopted a motif-based approach grounded in the Matrix Profile 
framework}. Prior to motif extraction, all series were normalized 
using z-score normalization applied independently to each time series, 
in order to focus the analysis on pattern shape rather than absolute 
scale. Motifs were extracted using the \texttt{STUMPY} implementation 
of the matrix profile. 
% A sliding window of length 28 was selected as a compromise between capturing meaningful temporal 
% dynamics and preserving local variations. 
A window length of 28 was selected to capture repeated seasonal 
dynamics rather than isolated cycles.
For each time series, the three subsequences associated with the 
lowest matrix profile values were selected, resulting in the extraction 
of the top-3 motifs per time series. \\
All extracted motifs were aggregated into 
a single dataset. To analyze similarities among 
motifs, K-Means clustering using Euclidean distance was applied. 
The number of clusters was selected by jointly evaluating the Sum 
of Squared Errors and the Silhouette Score, leading to the choice 
of four clusters as a reasonable trade-off between compactness and 
separation. Cluster centroids were analyzed as representative 
average patterns, and intra-cluster variability was quantified 
by computing the standard deviation across all motifs assigned to 
each cluster. Figure~\ref{fig:motif_clusters} shows the resulting 
cluster centroids together with their variability bands. In addition, 
clustering was also performed using Dynamic Time Warping as distance 
measure in order to account for possible temporal misalignments 
between motifs; this alternative formulation led to cluster 
structures and representative patterns that were largely consistent 
with those obtained using Euclidean distance. \\

\begin{figure}[H]
    \centering
    \begin{minipage}{0.45\textwidth}
        \centering
        \includegraphics[width=\textwidth]{plotsss/ts_motifs_kmeans.png} 
        \captionof{figure}{Motif clusters via K-Means.}
        \label{fig:motif_clusters}
    \end{minipage}
    \hfill
    \begin{minipage}{0.45\textwidth} 
        \includegraphics[width=\textwidth]{plotsss/ts_sax.png}
        \caption{SAX representation of motifs.}
        \label{ fig:sax_representation }
    \end{minipage}

\end{figure}

The two plots reveal three partially overlapping groups of motifs,
whose shapes are characterized by wave-like patterns with evolving
peaks:

\begin{itemize}
    \item \textbf{Cluster 0} (\textbf{cccb}) predominantly captures
    motifs from the earliest stages of the time series, characterized
    by a high initial peak followed by a gradual decline;

    \item \textbf{Cluster 2} (\textbf{cbbb}) corresponds mainly to
    early-to-middle stages of the time series, exhibiting moderate
    peaks and a smoother overall wave shape;

    \item \textbf{Cluster 3} (\textbf{baaa}) represents motifs from
    the mid-to-late stages of the time series, with a noticeably
    flattened waveform and lower-amplitude peaks.
\end{itemize}

The remaining SAX representation depicts motifs occurring in
later stages of the time series, which display an almost flat shape.
These motifs are therefore also assigned to \textbf{Cluster 3},
reinforcing its interpretation as capturing late-stage, low-variability
patterns.

The final cluster (\textbf{Cluster 1}), which is also the least
populated, exhibits a distinct behavior: its motifs show a clear
increasing trend, with rising peaks over time, setting it apart from
the predominantly decaying or flattening patterns observed in the
other clusters.